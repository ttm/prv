\documentclass{article}
\usepackage{graphicx}
\usepackage[usenames,dvipsnames]{xcolor}
\usepackage{hyperref}
\hypersetup{
	%pagebackref=true,
	pdfcreator={LaTeX with abnTeX2},
	pdfkeywords={abnt}{latex}{abntex}{USPSC}{trabalho acadêmico}, 
	colorlinks=true,       		% false: boxed links; true: colored links
	linkcolor=blue,          	% color of internal links
	citecolor=blue,        		% color of links to bibliography
	filecolor=magenta,      		% color of file links
	urlcolor=blue,
	allbordercolors=black,
	bookmarksdepth=4
}
\usepackage[utf8]{inputenc}
\usepackage{tipa}
% \usepackage{fontspec}
% \usepackage[mathletters]{ucs}
% \newcommand{\ttt}[1] {
% 	\texttt{<#1>}}
% \newcommand{\tttt}[1]{\texttt{#1}}

\begin{document}
\title{The Toki Pona Language:
an overview and some hacks}
% Anthropological computation
\author{Renato Fabbri\\
\texttt{renato.fabbri@gmail.com}\\
University of São Paulo,\\
Institute of Mathematical and Computer Sciences\\
São Carlos, SP, Brazil
}
\maketitle
\begin{abstract}

\end{abstract}

\section{Introduction}\label{intro}
Toki Pona is a minimalist conlang (constructed language)
with only 126 words.
% definition, historic, status (groups, speakers, etc).

\subsection{Resources on the web}
http://tokipona.net
\subsection{Historical note}

\subsection{Natural and constructed languages}
% conlangs, natural langs, programming langs, Lojban as an
% intermediary
\section{Overview of the language}\label{basics}
\subsection{Phonology}
Words in Toki Pona are written using only 14 letters:
\begin{itemize}
  \item Vowels a (open), e (mid front), o (mid back), i (close front),
    u (close back).
  \item Consonants j, k, l, m, n, p, s, t, w:
    \begin{itemize}
      \item Nasal: m (labial), n (coronal).
      \item Plosive: k (dorsal), p (labial), t (coronal).
      \item Fricative: s (coronal).
      \item Approximant: j (dorsal), l (coronal), w (labial).
    \end{itemize}
\end{itemize}

There are standard guidelines for pronunciation,
but the language allows for considerable allophonic
variation.
For example, /p t k s l/ might be pronounced
[p t k s l] or [b d g z \textipa{\!R}].

Syllables are of the form (C)V(N):
an optional consonant, a vowel and an optional nasal consonant.
Non word-initial syllables must follow the pattern CV(N).
The following sequences are forbidden: ji, wu, wo, ti, mn, nm, mm,
nn.

\subsection{Syntax}
As in other natural languages, 
colloquial Toki Pona might have
incomplete sentences and deviate from the
norm.
The basic structure of sentences are in the form:
<subject> li <predicate> e <object>.
The li might be repeated to associate more than
one predicate to the subject.
The particle li is omitted if the subject is a simple mi (I or us)
or sina (you). A discussion about problems with this rule
and how I deal with them is in Appendix~\ref{mytoki}.

The e might be repeated to associate more than
one object to a predicate.
Sentences might be related though la,
'sentence' la 'sentence', where the second sentence is
the main sentence, and the first sentence is a condition
to the first.
Multiple la-s are not described in literature,
but I assume that one might assume the last sentence
being a conditional to the next,
except in cases where the context strongly suggests
otherwise.

Noun and verb phrases are built with the non-particle words.
The first word is the noun and phrase and subsequent words
qualify the noun or phrase.
The pi particle might be used to separate sequences of words
to be evaluated before the relations yield by pi:
As pi is often ill understood and used,
the following structures might be handy for newbies and as a
reference:
\begin{itemize}
  \item No pi, `word word word':
word $\leftarrow$ (qualifies 1) word $\leftarrow$ (qualifies 2)  word.
  \item One pi, 'word pi word word': word $\leftarrow$ (qualifies 2) [
      word $\leftarrow$ (qualifies 1)  word ].
  \item Two pi-s: `word pi word word word pi word word':
    word $\leftarrow$5 [word 2 word] 3 word $\leftarrow$4 word 1  word;
or:
    word $\leftarrow$5 [word 1 word] 2 word $\leftarrow$4 word 3  word.
\end{itemize}

Notes on the usage of pi:
\begin{itemize}
  \item In a sequence of words, without pi, the second word qualifies
    the first, the third word qualifies the phrase yield by the first
    two words, the fourth word qualifies the noun yield by the first
    three words and so on.
  \item It is redundant to use pi before the last word in a noun or
    verb phrase if there is no other pi, reason why it is most often
    omitted.
    Its use in this case is regarded as wrong~\cite{lipu,pije},
    but, as one might notice, it does not introduce any ambiguity.
  \item The book by jan Pije~\cite{kama} describes another use for pi:
    after li to mean possession, e.g. `soweli li pi sina' (your pet).
    This employment of pi might be regarded as correct, but are promptly written
    as a noun phrase (e.g. `soweli sina') and is not mentioned
    by the official book~\cite{tpLabg}.
\end{itemize}

All the words except the structural particles (li, e, la, pi)
are usable in noun and verb phrases.
Notice that the phrase expresses a noun in a noun phrase (subject or
object) or a verb (in the predicate).

At this point, the only missing syntax rule is related to
the prepositions: kepeken, lon, sama, tan, tawa.
They might appear at the end of noun phrases,
should be followed by  another noun phrase,
and require no particle. E.g.
`toki tan jan Pije li pana e sona tawa mi'.

Other particles are: a or kin, o, taso, anu, en, 
nanpa, seme. mu?, 

Vocatives should have an o after the noun phase.

\subsection{Further notes}
The only synonyms on Toki Pona are:
a and kin; lukin and oko;
sin or kamako;
ale or ali.

toki e ni, wile e ni, etc.
'e ni' can be omitted and : used by itself.

Names are by default transliterated,









% my view
% considerations about the official book and jan Pije
% pictorial forms
\subsection{Main references for the language}
\begin{itemize}
  \item The official book is ``Toki Pona: The Language of Good''
    and is authored by Sonja Lang, the creator of the language.
  \item The book ``o kama sona e toki pona!'', from jan Pije,
    is the other main reference for the language~\cite{kama}.
  \item \cite{Wikipesija}.
\end{itemize}

\section{Analysis and hacks for the language}\label{hacks}
% number of possible syllables
% number of possible words of size X (1, 2, 3, 4....).
% syntax highlighting, change colors on the fly
% analysis of the language
% synthesis of sentences and texts
% hacks from other ppl
\subsection{Hacks from other people}
The \url{tokipona.net} has a number of tools,
just as to transliterate names into Toki Pona phonemes,
and search in corpus.

\section{Conclusions and further work}\label{conc}
\begin{itemize}
  \item Relate Toki Pona to Wordnet: should one Toki Pona word
    be related to more then one synset of the English language?
  \item Understand how the corpus is gathered in \url{tokipona.net}.
  \item Know about previously existing words that were used for Toki Pona
    (e.g. suno and suwi might come from sun and sweet),
    and about the reasons that lead Sonja (and maybe other people)
    to choose the 14 letters and the syllable structure.
    This might require a dedicated communication with the
    speaker community and the documentation authors.
  \item Corpus-based analysis.
  \item Publication of original texts and translations.
\end{itemize}
\subsection*{Acknowledgments}
FAPESP (project 2017/05838-3); Vim developers and documentation maintainers;
Vim user community. 

\appendix
\section{My usage of Toki Pona}\label{mytoki}
I use the standard sounds, but often use [z] for s.
I often translate texts to Toki Pona (e.g. biblical excepts)
and create new texts as poems and short stories.
Most of them are in~\cite{tokisona}.
I omit the li particle after subjects sina and mi,
in accordance with the norm,
but sometimes I use them when there are many predicates.
E.g. sina li wawa li pimeja li lukin pona li moku e kasi mute.
In such cases, the first li is sometimes omitted.
Also, sometimes I use li before mi and sina where I find
that there is unwanted ambiguity, e.g.
sina moku pona e jan 
(might be sina li moku pona e jan or sina moku li pona e jan).

Names are by default transliterated,
but I advocate that, as in other languages,
names might be used as they are in the
correspondent mother tongue.
E.g. the name Erdös is used in
English and Portuguese although the standard
alphabet does not contain ö in such languages.
I also tend to legitimate the use of English (or German) words
in Toki Pona texts if it is the case,
as happens often in scientific writing
(kernel is a German word used in English,
webpage is an English word used in Portuguese).



\begin{thebibliography}{99}
\fontsize{11}{0}\selectfont
\bibitem{anPh}
	Anthropological physics and social psychology in the critical research of networks. Complex Networks Digital Campus (CS-DC'15).
	Available at \url{https://youtu.be/oeOKYc3-nbM}
\bibitem{anPh2}
	Fabbri, R. What are you and I? [anthropological physics fundamentals], 2015. Available at \url{https://www.academia.edu/10356773/What_are_you_and_I_anthropological_physics_fundamentals_}
\bibitem{ouroWiki}
  Ouroboros. (2017, November 2). In Wikipedia, The Free Encyclopedia. Retrieved 22:19, November 9, 2017, from \url{https://en.wikipedia.org/w/index.php?title=Ouroboros&oldid=808392809}
\bibitem{vimrc}
	Fabbri, R. (2017). A reasonable vimrc file. Available at \url{https://raw.githubusercontent.com/ttm/vim/master/vimrc} 
\bibitem{ex}
  Ex (text editor). (2017, March 22). In Wikipedia, The Free Encyclopedia. Retrieved 22:22, November 9, 2017, from \url{https://en.wikipedia.org/w/index.php?title=Ex_(text_editor)&oldid=771621020}
\bibitem{tokipona}
	Fabbri, R. (2017). A Toki Pona Python Package and Vim Syntax Highlighting. Available at \url{https://github.com/ttm/tokipona} 
\bibitem{tpLang}
	Lang, S. (2014). Toki Pona: the language of good. Tawhid Publishing.
    ISBN-10: 0978292308, ISBN-13: 978-0978292300.
\bibitem{aa1}
	Fabbri, R., Fabbri, R., Vieira, V., Penalva, D., Shiga, D., Mendonça, M., Negrao, A., Zambianchi, L., \& Thumé, G. (2013). AA: The Algorithmic Autoregulation (Distributed Software Development) Methodology. RESI. From \url{https://arxiv.org/abs/1604.08255}
\bibitem{aa2}
	Fabbri, R. (2017).
The Algorithmic-Autoregulation (AA) Methodology and Software:
a collective focus on self-transparency. ENMC2017. From \url{https://github.com/ttm/ensaaio/raw/master/emc/article.pdf} 
\bibitem{kama}
  Knight, B. (2017) o kama sona e toki pona! Available at:
    \url{http://tokipona.net/tp/janpije/okamasona.php}
\bibitem{Wikipesija}
  Toki Pona community (2017). Wikipesija. Available at:
    \url{http://tokipona.wikia.com}
\end{thebibliography}
\end{document}
