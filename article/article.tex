\documentclass{article}
\usepackage{graphicx}
\usepackage[usenames,dvipsnames]{xcolor}
\usepackage{hyperref}
\hypersetup{
	%pagebackref=true,
	pdfcreator={LaTeX with abnTeX2},
	pdfkeywords={abnt}{latex}{abntex}{USPSC}{trabalho acadêmico}, 
	colorlinks=true,       		% false: boxed links; true: colored links
	linkcolor=blue,          	% color of internal links
	citecolor=blue,        		% color of links to bibliography
	filecolor=magenta,      		% color of file links
	urlcolor=blue,
	allbordercolors=black,
	bookmarksdepth=4
}
\usepackage[utf8]{inputenc}
\usepackage{tipa}
% \usepackage{fontspec}
% \usepackage[mathletters]{ucs}
% \newcommand{\ttt}[1] {
% 	\texttt{<#1>}}
% \newcommand{\tttt}[1]{\texttt{#1}}

\begin{document}
\title{The Toki Pona Language:
an overview and some hacks}
% Anthropological computation
\author{Renato Fabbri\\
\texttt{renato.fabbri@gmail.com}\\
University of São Paulo,\\
Institute of Mathematical and Computer Sciences\\
São Carlos, SP, Brazil
}
\maketitle
\begin{abstract}

\end{abstract}

\section{Introduction}\label{intro}
Toki Pona is a minimalist conlang (constructed language)
with only 126 words.
% definition, historic, status (groups, speakers, etc).

\subsection{Resources on the web}
http://tokipona.net
\subsection{Historical note}

\subsection{Natural and constructed languages}
% conlangs, natural langs, programming langs, Lojban as an
% intermediary
\section{Overview of the language}\label{basics}
\subsection{Phonology}
Words in Toki Pona are written using only 14 letters:
\begin{itemize}
  \item Vowels a (open), e (mid front), o (mid back), i (close front),
    u (close back).
  \item Consonants j, k, l, m, n, p, s, t, w:
    \begin{itemize}
      \item Nasal: m (labial), n (coronal).
      \item Plosive: k (dorsal), p (labial), t (coronal).
      \item Fricative: s (coronal).
      \item Approximant: j (dorsal), l (coronal), w (labial).
    \end{itemize}
\end{itemize}

There are standard guidelines for pronunciation,
but the language allows for considerable allophonic
variation.
For example, /p t k s l/ might be pronounced
[p t k s l] or [b d g z \textipa{\!R}].

Syllables are of the form (C)V(N):
an optional consonant, a vowel and an optional nasal consonant.
Non word-initial syllables must follow the pattern CV(N).
The following sequences are forbidden: ji, wu, wo, ti, mn, nm, mm,
nn.

\subsection{Syntax}
As in other natural languages, 
colloquial Toki Pona might have
incomplete sentences and deviate from the
norm.
The basic structure of sentences are in the form:
<subject> li <predicate> e <object>.
The li might be repeated to associate more than
one predicate to the subject.
The particle li is omitted if the subject is a simple mi (I or us)
or sina (you). A discussion about problems with this rule
and how I deal with them is in Appendix~\ref{mytoki}.

The e might be repeated to associate more than
one object to a predicate.
Sentences might be related though la,
'sentence' la 'sentence', where the second sentence is
the main sentence, and the first sentence is a condition
to the first.
Multiple la-s are not described in literature,
but I assume that one might assume the last sentence
being a conditional to the next,
except in cases where the context strongly suggests
otherwise.

Noun and verb phrases are built with the non-particle words.
The first word is the noun and phrase and subsequent words
qualify the noun or phrase.
The pi particle might be used to separate sequences of words
to be evaluated before the relations yield by pi:
As pi is often ill understood and used,
the following structures might be handy for newbies and as a
reference:
\begin{itemize}
  \item No pi, `word word word':
word $\leftarrow$ (qualifies 1) word $\leftarrow$ (qualifies 2)  word.
  \item One pi, 'word pi word word': word $\leftarrow$ (qualifies 2) [
      word $\leftarrow$ (qualifies 1)  word ].
  \item Two pi-s: `word pi word word word pi word word':
    word $\leftarrow$5 [word 2 word] 3 word $\leftarrow$4 word 1  word;
or:
    word $\leftarrow$5 [word 1 word] 2 word $\leftarrow$4 word 3  word.
\end{itemize}

Notes on the usage of pi:
\begin{itemize}
  \item In a sequence of words, without pi, the second word qualifies
    the first, the third word qualifies the phrase yield by the first
    two words, the fourth word qualifies the noun yield by the first
    three words and so on.
  \item It is redundant to use pi before the last word in a noun or
    verb phrase if there is no other pi, reason why it is most often
    omitted.
    Its use in this case is regarded as wrong~\cite{lipu,pije},
    but, as one might notice, it does not introduce any ambiguity.
  \item The book by jan Pije~\cite{kama} describes another use for pi:
    after li to mean possession, e.g. `soweli li pi sina' (your pet).
    This employment of pi might be regarded as correct, but are promptly written
    as a noun phrase (e.g. `soweli sina') and is not mentioned
    by the official book~\cite{tpLabg}.
\end{itemize}

All the words except the structural particles (li, e, la, pi)
are usable in noun and verb phrases.
Notice that the phrase expresses a noun in a noun phrase (subject or
object) or a verb (in the predicate).

At this point, the only missing syntax rule is related to
the prepositions: kepeken, lon, sama, tan, tawa.
They might appear at the end of noun phrases,
should be followed by  another noun phrase,
and require no particle. E.g.
`toki tan jan Pije li pana e sona tawa mi'.

Other particles are:
\begin{itemize}
  \item a or kin, emphasis.
  \item o, vocative or imperative ('jan lukin sitelen o, li wawa')
  \item taso, means however or 'only' if adjective.
  \item anu, en: 'or' and 'and'. Used for nouns in nous phrases.
    For repetition of verbs, repeat li.
    For object nouns, repeat e.
    If the noun is complementing a preposition (tawa, lon),
    one might repeat the preposition.
    As Toki Pona is a recent language, and is able to cope with
    variation due to its simplicity, I would advocate for
    using en and anu wherever there is no ambiguity.
  \item nanpa, denotes numbering.
  \item seme, for questions, used next to the thing being asked for.
    'Why?' might be expressed as
    'seme la sina pana e moku lon sewi'.
  \item mu, for animal noises. For me it is not a particle, as in the
    official dictionary, but a noun.
    I also like to use it as a verb:
    'mi pakala e luka. mu mute.
\end{itemize}

The vocabulary specifies morphosyntactic classes:
nouns, adjectives, verbs, pre-verbs, adverbs, particles, prepositions, and numbers.
I find that they might help the user and newcomer, but
it might also suggest a deviation from what I understand and read:
the words might be used indistinctly to be the nouns
(subjects, the predicate when there is no object, objects,
and preposition complements),
the adjectives (anything that does not start the noun phrase or follows a pi),
verbs
(follows mi or sina or li or a preposition),
adverbs (follows the verb).
The pre-verbs (wile, ken, awen, kama, lukin, sona),
might follow a verb, but might also be understood
as the verb qualified with the next word,
which carries a very similar if not identical meaning.
The pre-verbs are all also defined as other morphosyntactic classes,
such as adjective, noun, verb.
The only exception is wile, which is only a pre-verb.

Thus, the classes given in the dictionary dictate little
in practice:
jan kala li lape lon ni.
Where kala, lape and ni are in this phrase
as adjective, verb and noun,
and are in the dictionary as noun,
adjective and adjective.

As far as I can see, one should regard
the particles li, e, pi, and la
and punctuation.
The other tokens of the vocabulary
might be used in any of the remaining positions.
Detection:
\begin{itemize}
  \item Noun: the first word in a noun phrase.
    After an 'e' and after a pi,
    The first word in the sentence if
    sentence does not start by the verb.
    Might be in the position of the verb
    if the sentence has no object.
  \item Adjective: second word on after an e and after
    Second word on in the subject phrase if present.
  \item Verb: after a li, mi or sina.
    If there is no object, the verb position
    is often a noun.
  \item After a preposition,
    there can be a noun phrase, a verb phrase
    or nothing.
  \item Notice that there is ambiguity in the structure
    introduced by the omission of li after mi and sina.
    Also, when there is no object, a noun or a verb
    or an adjective might be in the verb position
    if there is no object.
    The prepositional complement is also not defined.
    (mi moku tawa pali, tawa tomo).
    So, these are sources of syntactic ambiguities
    in Toki Pona.
    They might be solved or minimized by using the semantics
    of the words.
    One preliminary effort in this direction might be
    using the classes in the official dictionary to resolve
    ambiguities whenever possible.
    This solution is not optimal in correct POS tagging,
    and does not solve all possible ambiguities
    (there are words classified as nouns and adjectives,
    adjectives and verbs, nouns and verbs, particle and verb).

    Another source of ambiguity is the pre-verbs as described
    in the literature~\cite{tpLang,pije}.
    But I find it reasonable to understand them as verbs.

\end{itemize}

Also, the prepositional complement might
be a noun or a verb.
I could not come up with a sentence
where it would be understood as an adjective.





\subsection{Further notes}
The only synonyms on Toki Pona are:
a and kin; lukin and oko;
sin or kamako;
ale or ali.

In formations such as
toki e ni:, wile e ni:, tan ni: etc.
'(e) ni' can be omitted and : used alone.

Names are by default transliterated,
but might not be, as described in Section~\ref{mytoki}.


% my view
% considerations about the official book and jan Pije
% pictorial forms
\subsection{Main references for the language}
\begin{itemize}
  \item The official book is ``Toki Pona: The Language of Good''
    and is authored by Sonja Lang, the creator of the language.
  \item The book ``o kama sona e toki pona!'', from jan Pije,
    is the other main reference for the language~\cite{kama}.
  \item \cite{Wikipesija}.
\end{itemize}

\section{Analysis and hacks for the language}\label{hacks}

\section{Statistics of the vocabulary}
In~\cite{pije} there is statistics about Toki Pona corpus.
This section focuses on the statistics of the vocabulary
and syntactic rules:
the letters, phonemes, word sizes,
possible combinations for words and sentences.
The statistics are in Appendix~\cite{listings},
and the next paragraph is an overview.
Python scripts were used to obtain the measurements
and are available at~\cite{tokipona}.

As described in Section~\cite{basics},
there is only 14 letters,
and phonemes respect a few rules.
There are 120 different words in the official vocabulary,
4 of them having synonyms.
A total of 124 tokens.
Not counting proper nouns (names)
and punctuation.
They include X nouns, Y verbs, Z adjectives,
W prepositions, Y particles, K pre-verbs.
Of which one might distinguish only between
the particles and the other J words that
might be used anywhere a particle will not.

Most often vowels,
consonants, consonant vs vowel.
Most often phonemes in general an specific positions..
Possible phonemes given the rules.
Comparison of the occurring against the possible.

Syntactic structures:
Possibilities on noun, verb and prepositional phrases
and in sentences.
Counting along the number of words.


\section{Synthesis of text}
The same package~\cite{tokipona}
has capabilities for synthesizing text.
Noun, verb and prepositional phrases,
sentences.
It also aims at making larger scale texts
by keeping a record of the used words and structures (context)
and using stylistic outlines for poems and short narratives.

% number of possible syllables
% number of possible words of size X (1, 2, 3, 4....).

\section{Syntax highlighting}
The same package~\cite{tokipona}
has a Vim syntax highlighting plugin
for Toki Pona.
Instructions for installing and using
the syntax highlighting is at~\cite{tokipona}.

Basically, it distinguishes the words among the morphosyntactic
classes according to the official dictionary.
As a word often belongs to more than one class,
The precedence of them might be set by the user.
Also, some classes might be further refined or joined,
such as by distinguishing only particles and the rest,
or maybe particles and prepositions and the rest.
The colors are also promptly changed according to~\cite{vimArt}
and exemplified in the package documentation.

Currently, the Python package synthesizes the
syntax file.
The user has control of class precedence and
merging.
The choice of precise coloring schemes
might involve hacking the colorscheme being
used in Vim (such as 'blue', 'elflord' and 'gruvbox'),
and Vim's highlighting schemes as described in~\cite{vimArt}.
In summary,
the usage of the package and plugin might be performed
through the following actions:
\begin{itemize}
  \item Installation of the plugin.
  \item Tweak of the syntax file by hand.
  \item Running the Python script to generate a new syntax file
    according to other settings.
  \item Write a file inside Vim using Toki Pona and save
    the file with the .tokipona extension.
    Reload the highlighting scheme whenever you
    change the syntax file by hand or through the Python script.
  \item Access the used highlighted groups with :syntax,
    Access all the highlighting groups with :so \$VIMRUNTIME/syntax/hitest.vim.
    Change the coloring of a set of terms by associating
    a used group (e.g. tokiponaADJECTIVE) to an existing group (e.g. Visual):
    :highlight link tokiponaADJECTIVE Visual.
    The plugin comes with the :TokiStation command, which opens a window with
    the files: ijositelen.tokipona, makevimSyntax.py, Highlight Test (created by hitest.vim above), syntax/tokipona.vim.
    Another tab with shells: python to run over the makeVimSyntax.py and make new syntax/tokipona.vim files. Another with Readme, and PDF documentation. Another with an IPython.
    Imported tokipona. It resets completely upon command :TokiClose, closing all created windows.
\end{itemize}

\subsection{Hacks from other people}
The \url{tokipona.net} has a number of tools,
just as to transliterate names into Toki Pona phonemes,
and search in corpus.

\section{Conclusions and further work}\label{conc}
\begin{itemize}
  \item Relate Toki Pona to Wordnet: should one Toki Pona word
    be related to more then one synset of the English language?
  \item Understand how the corpus is gathered in \url{tokipona.net}.
  \item Know about previously existing words that were used for Toki Pona
    (e.g. suno and suwi might come from sun and sweet),
    and about the reasons that lead Sonja (and maybe other people)
    to choose the 14 letters and the syllable structure.
    This might require a dedicated communication with the
    speaker community and the documentation authors.
  \item Corpus-based analysis.
  \item Publication of original texts and translations.
  \item Make an article written in Toki Pona.
    I wrote Section~\ref{finalToki}, and believe that
    a summary in both English and Toki Pona for
    facilitating the acquisition of context,
    a reasonable article on some scientific topics is
    possible.
    I first conceived something around complexity, statistics, physics, or computer science.
    But one possibility is to write about linguistics, philosophy, literature
    or psychology with the partners I write in English.
    I can start a draft,
    they might learn the language in a few hours (with or without my help),
    to contribute, and we can write a short paper.
  \item Enhance the synthesis of text to yield better contextualized text
    and stylistic traces, such as for poems and short stories.
    Give the user the ability to choose the sentences (generate randomly according to
    previously written or given text,
    some rules input by the user, the package and the language guidelines,
    outputs to the screen and asks to keep and discard).
  \item Enhance the syntax highlighting implementation described in Section~\ref{high}.
    It uses only syntactic cues to choose POS tags,
    which might be overcome by using n-grams and
    further techniques from Natural Language Processing~\cite{POS}.
\end{itemize}

\subsection*{Acknowledgments}
FAPESP (project 2017/05838-3); Vim developers and documentation maintainers;
Vim user community. 

\appendix
\section{My usage of Toki Pona}\label{mytoki}
I use the standard sounds, but often use [z] for s.
I often translate texts to Toki Pona (e.g. biblical excepts)
and create new texts as poems and short stories.
Most of them are in~\cite{tokisona}.
I omit the li particle after subjects sina and mi,
in accordance with the norm,
but sometimes I use them when there are many predicates.
E.g. sina li wawa li pimeja li lukin pona li moku e kasi mute.
In such cases, the first li is sometimes omitted.
Also, sometimes I use li before mi and sina where I find
that there is unwanted ambiguity, e.g.
sina moku pona e jan 
(might be sina li moku pona e jan or sina moku li pona e jan).

Names are by default transliterated,
but I advocate that, as in other languages,
names might be used as they are in the
correspondent mother tongue.
E.g. the name Erdös is used in
English and Portuguese although the standard
alphabet does not contain ö in such languages.
I also tend to legitimate the use of English (or German) words
in Toki Pona texts if it is the case,
as happens often in scientific writing
(kernel is a German word used in English,
webpage is an English word used in Portuguese).

Proposed notations for numbers seem numerous.
I tend to think that one might indicate is two numbers
are multiplying (pi) or are in different scales
(such as in decimal or binary notations).
For example, I take luka two to mean 52.
Or it might not be taken to such level of strictidness,
for 52 is a reasonable notation for a simple language.
mi jo e jan sama nanpa 12.
Or even:
ona li lon e soweli 27.

Remove the 'li'.
I've been avoiding e ni: and using only :.
've been omitting the subject if it is the
same as in the last sentence.
I've been avoiding also the li sometimes,
and starting only a predicate + object phrase,
or a whole phrase altogether.

\begin{table*}[h!]
\begin{center}
\caption{"POS tags incident and chosen as preferential e.g. in text synthesis.
            The official dictionary often relates tokens
            to more than one POS tag.
            For the text highlighting Plugin, for example,
            a token has to have an established tag to have
            a defined color.
            On the Chosen column, the tokens were regarded only once
            by choosing the first occurrence of ['PRE', 'VERB', 'PREPOSITION', 'PARTICLE', 'ADJECTIVE', 'NOUN', 'NUMBER']
            in the official dictionary.}\label{foobar}
\begin{tabular}{| l | c | c |}\hline
POS & All  & Chosen \\\hline
NOUN & 58  & 49 \\\hline
ADJECTIVE & 40  & 34 \\\hline
VERB & 15  & 13 \\\hline
PARTICLE & 12  & 12 \\\hline
PRE & 6  & 6 \\\hline
PREPOSITION & 5  & 5 \\\hline
NUMBER & 4  & 1 \\\hline
total & 140  & 120 \\\hline
\end{tabular}\end{center}
\end{table*}



\section{Final words in Toki Pona}
toki li nasin e lawa.
li nasin tawa (pi) toki insa.

% jan li ken toki li ken pali pona e toki pona.
toki pona li pona e nasin tan ni:
ona li jo e sitelen mute lili.
ona li pona.
pona kepeken weka 'p' li ona, a.

o taso la toki pona li kalama li lukin pona.
sitelen en nasin li open e sitelen suli\cite{tpLang,pije,fb1,fb2,tokisona,Wikipesija}.
li sona. li nasin e toki e sona e lawa e lon.

toki ni li wawa tawa jan mute nasin en toki.
wawa tawa toki pi jan sona.
pi ilo nanpa en nanpa nasin.
taso tawa toki sona a.
sitelen sona, sitelen musi.

ilo lon linha~\ref{hacks} li pana e sitelen
e sona tan sitelen,
en nasin.
linha~\ref{basics} li pana e nasin pi toki pona.
e sitelen tawa kama sona.

mi wile pali e sitelen lon toki pona.
sitelen lon nasin, sitelen sona,
sitelen musi.
ante e toki pona la
la ona li nasa.
taso nasa li pona mute.
li pona mute tawa lawa,
tawa kama sona, tawa sitelen e toki.
ante la toki pona o.
toki e toki pona tawa sina.
kama sona e toki pona sina.

o pona tawa jan pi toki pona.
jan Sonja, Birns-Sprage, Kipo, Pije,
Siwejo, Malija, jan kulupu mute.

\begin{thebibliography}{99}
\fontsize{11}{0}\selectfont
\bibitem{anPh}
	Anthropological physics and social psychology in the critical research of networks. Complex Networks Digital Campus (CS-DC'15).
	Available at \url{https://youtu.be/oeOKYc3-nbM}
\bibitem{anPh2}
	Fabbri, R. What are you and I? [anthropological physics fundamentals], 2015. Available at \url{https://www.academia.edu/10356773/What_are_you_and_I_anthropological_physics_fundamentals_}
\bibitem{ouroWiki}
  Ouroboros. (2017, November 2). In Wikipedia, The Free Encyclopedia. Retrieved 22:19, November 9, 2017, from \url{https://en.wikipedia.org/w/index.php?title=Ouroboros&oldid=808392809}
\bibitem{vimrc}
	Fabbri, R. (2017). A reasonable vimrc file. Available at \url{https://raw.githubusercontent.com/ttm/vim/master/vimrc} 
\bibitem{ex}
  Ex (text editor). (2017, March 22). In Wikipedia, The Free Encyclopedia. Retrieved 22:22, November 9, 2017, from \url{https://en.wikipedia.org/w/index.php?title=Ex_(text_editor)&oldid=771621020}
\bibitem{tokipona}
	Fabbri, R. (2017). A Toki Pona Python Package and Vim Syntax Highlighting. Available at \url{https://github.com/ttm/tokipona} 
\bibitem{tpLang}
	Lang, S. (2014). Toki Pona: the language of good. Tawhid Publishing.
    ISBN-10: 0978292308, ISBN-13: 978-0978292300.
\bibitem{aa1}
	Fabbri, R., Fabbri, R., Vieira, V., Penalva, D., Shiga, D., Mendonça, M., Negrao, A., Zambianchi, L., \& Thumé, G. (2013). AA: The Algorithmic Autoregulation (Distributed Software Development) Methodology. RESI. From \url{https://arxiv.org/abs/1604.08255}
\bibitem{aa2}
	Fabbri, R. (2017).
The Algorithmic-Autoregulation (AA) Methodology and Software:
a collective focus on self-transparency. ENMC2017. From \url{https://github.com/ttm/ensaaio/raw/master/emc/article.pdf} 
\bibitem{kama}
  Knight, B. (2017) o kama sona e toki pona! Available at:
    \url{http://tokipona.net/tp/janpije/okamasona.php}
\bibitem{Wikipesija}
  Toki Pona community (2017). Wikipesija. Available at:
    \url{http://tokipona.wikia.com}
\end{thebibliography}
\end{document}
