\documentclass{article}
\usepackage{graphicx}
\usepackage[usenames,dvipsnames]{xcolor}
\usepackage{hyperref} 	%% Use to fix Figure or Table: ex: \begin{table}[H]
\hypersetup{
	%pagebackref=true,
	pdfcreator={LaTeX with abnTeX2},
	pdfkeywords={abnt}{latex}{abntex}{USPSC}{trabalho acadêmico}, 
	colorlinks=true,       		% false: boxed links; true: colored links
	linkcolor=blue,          	% color of internal links
	citecolor=blue,        		% color of links to bibliography
	filecolor=magenta,      		% color of file links
	urlcolor=blue,
	allbordercolors=black,
	bookmarksdepth=4
}
\usepackage[utf8]{inputenc}

\begin{document}
\title{On the Vim editor:\\
vim features and tweaks after 10 years of usage}

\author{Renato Fabbri\\
\texttt{renato.fabbri@gmail.com}\\
University of São Paulo,\\
Institute of Mathematical and Computer Sciences\\
São Carlos, SP, Brazil
}
\maketitle
\begin{abstract}
The Vim editor is very rich in capabilities
and thus complex.
This article is a description of the Vim text editor
and a set of enhancements proposed for it.
It is the result of more than ten years of experience
in using Vim for writing and editing various types of documents,
mostly:
Python, C++, JavaScript, ChucK, etc, programs;
and \LaTeX, Markdown, HTML, RDF, Make and other markup files;
binary files.
It is said that it takes about ten years to master this
text editor, and I find that other experienced users
have a different view of Vim and that they use a different
set of tools.
Therefore, this document exposes my insights in order
to confront my usage with the praxis of the Vim users community
and to make available a reference document with which new users
can grasp an overview by reading it and the discussions that it may generate.
Also, it should be useful for users of any degree of experience,
including me, as a compendium of commands, namespaces and tweaks.
Upon feedback, and maturing of my Vim usage,
this document might be enhanced or receive additional
material.
\end{abstract}

\section{Introduction}
Vim is a very complex editor,
considered by the Linux community to be
matched only by Emacs.
They both are the standard advanced text editors
of the open source community
and have beed developed for decades.
This document describes the Vim editor
and proposes a set of enhancements of the user
experience which
are implemented as Vim Plugins.
The editor has a very mature documentation
and the contents herein presented is a
report on the overall understandings I
have of Vim after a little bit more
of ten years using it.
The purposes of this document are:
\begin{itemize}
  \item to help new users in grasping Vim essentials
  and convenient practices.
  \item To record the view of the editor that
  an user (me) has after $\approx 10$ years of usage.
  \item To confront my usage with that of other experienced
  users. This is helpful for me, but also for the other users
  as they might benefit from this content and from discussions
  that might arise from it.
\end{itemize}

I used other editors, e.g. Kate, gedit, and Notepad2.
I used Vim for writting and editing computer code (Python, Javascript, C++, ChucK, bash, etc), markup languages (HTML, CSS, RDF, Markdown, \LaTeX, etc) and binary files.
Eventually, I edited database files and other types os files.
With Vim, I mostly write software (web and scientific),
music, poems and short stories.
It is very useful because:
\begin{itemize}
  \item it is meant to be a plain text (e.g. ascii, utf8) editor
  and does not (by standard) insert special charaters (e.g. for formating, with binary instructions).
  \item It has a powerful architecture and set of commands.
  \item It is highly configurable and most often the users
  has a set of commands.
\end{itemize}

The standard capabilities of the editor
should become clear in Section~\ref{basics}.
Vim is constantly evolving and there are many plugins,
some of them very popular for both general and specific
type of users.
Accordingly, there are many possible enhancements,
and Section~\ref{issues} report the most prominent of them
for me and potential workarounds made available as plugins.

\subsection{Historical note}
Vim was first released publicly in 1991.
It is a cross-platform GNU licensed free and open source extended clone of Bill Joy's vi text editor.

% Current stable, alpha, beta and development version
% no articles on Vim


\section{Basics}\label{basics}
Vim's interface is text-based.
In the GUI mode (gVim),
there are convenient menus and toolbars
but all functionalities are still available though
the command line mode.
Vimscript is the internal language of Vim,
and is often used for scripting by users
although other languages might be used 
(e.g. Python, Perl, Lua, Racker, Ruby and Tcl). 

Vim has some basic modes of usage:
\begin{itemize}
  \item Navigation mode: used for changing
  the position of the cursor or the text displayed
  at the window.
  A code goal of the navigation mode it to allow fast
  navigation of the document while allowing
  the typist to maintain its fingers on the home row
  (i.e. on the center of the keyboard).
  \item Insert mode:
  \item Command mode:
  \item Ex mode:
\end{itemize}

\subsection{Help commands, files, tutor, and usr\_toc}

\subsection{Using Vim's modes}
\subsubsection{Navigating}
hjkl keys
crtl+[dubf]
jumps
markers
Most important help files on Netrw

\subsubsection{Insertion}
ctrl+o

\subsection{Navigation + insertion}
Commands that bridge from Navigation to Insertion:
csrCSR
Default, line and block visual selection:
  and AiIcCdD

\subsection{Netrw}
No insert mode.

Edit directory
Sexplore
Create and delete files and directories
open files: on window, split, preview
Change display of files
Bookmarks
Most important help files on Netrw


\subsection{Standard configuration files and directories and My .vim/vimrc}
\subsection{Spell and spelllang (en and pt\_br)}
\subsection{Tabs, splits, buffers and namespaces}
\& \% \$ and the following
\subsection{Mappings and abbreviations}
\subsection{Macros and registers for copy and paste}
\subsection{History of commands}
\subsection{List of markers}
\subsection{Undo}
\subsection{Scripting, Functions, Vimscript and Python}
\subsection{Plugins}
\subsubsection{Standard features}
\subsubsection{Writting plugins}
\subsubsection{Plugin systems: usage for using and writing plugins}
\subsection{Colour}
\subsubsection{Standard, 8 and 16 bits, and true color, Screen/Byobu}
\subsubsection{Gruvebox, Solarize and other colourschemes}
\subsection{Fonts}
\subsubsection{Cools nd popular fonts}
\subsubsection{How to set fonts in xterm, gnome-terminal and GVim.}
\subsection{Verbosing, logs and possibilities of using it to study your own}
\subsection{age (often used commands and typed sequences).}
\subsection{Highlighting}
\subsection{Bash and Vim commands}
\subsection{Compiling, standard features and plugins}
\subsection{Quickfix}
\section{Issues and plugins}\label{issues}

\section{Final words and further work}

Potential enhancements to this document are:
\begin{itemize}
  \item The inclusion of reading emails and connecting over ssh.
\end{itemize}

\end{document}
